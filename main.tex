%%
%% This is file `sample-sigconf.tex',
%% generated with the docstrip utility.
%%
%% The original source files were:
%%
%% samples.dtx  (with options: `sigconf')
%% 
%% IMPORTANT NOTICE:
%% 
%% For the copyright see the source file.
%% 
%% Any modified versions of this file must be renamed
%% with new filenames distinct from sample-sigconf.tex.
%% 
%% For distribution of the original source see the terms
%% for copying and modification in the file samples.dtx.
%% 
%% This generated file may be distributed as long as the
%% original source files, as listed above, are part of the
%% same distribution. (The sources need not necessarily be
%% in the same archive or directory.)
%%
%% The first command in your LaTeX source must be the \documentclass command.
\documentclass[sigconf,review]{acmart}

\usepackage{code}
\usepackage{graphicx}
\usepackage{balance}

%%% The following is specific to Onward! '21 and the paper
%%% 'Let a Thousand Flowers Bloom: On the Uses of Diversity in Software Testing'
%%% by Alex Groce.
%%%
\setcopyright{acmcopyright}
\acmPrice{15.00}
\acmDOI{10.1145/3486607.3486772}
\acmYear{2021}
\copyrightyear{2021}
\acmSubmissionID{onward21essays-id2-p}
\acmISBN{978-1-4503-9110-8/21/10}
\acmConference[Onward! '21]{Proceedings of the 2021 ACM SIGPLAN International Symposium on New Ideas, New Paradigms, and Reflections on Programming and Software}{October 20--22, 2021}{Chicago, IL, USA}
\acmBooktitle{Proceedings of the 2021 ACM SIGPLAN International Symposium on New Ideas, New Paradigms, and Reflections on Programming and Software (Onward! '21), October 20--22, 2021, Chicago, IL, USA}


%%
%% Submission ID.
%% Use this when submitting an article to a sponsored event. You'll
%% receive a unique submission ID from the organizers
%% of the event, and this ID should be used as the parameter to this command.
%%\acmSubmissionID{123-A56-BU3}

%%
%% The majority of ACM publications use numbered citations and
%% references.  The command \citestyle{authoryear} switches to the
%% "author year" style.
%%
%% If you are preparing content for an event
%% sponsored by ACM SIGGRAPH, you must use the "author year" style of
%% citations and references.
%% Uncommenting
%% the next command will enable that style.
%%\citestyle{acmauthoryear}

%%
%% end of the preamble, start of the body of the document source.
\begin{document}

%%
%% The "title" command has an optional parameter,
%% allowing the author to define a "short title" to be used in page headers.
\title{Looking for Lacunae in Bitcoin Core's Fuzzing Efforts}

%%
%% The "author" command and its associated commands are used to define
%% the authors and their affiliations.
%% Of note is the shared affiliation of the first two authors, and the
%% "authornote" and "authornotemark" commands
%% used to denote shared contribution to the research.
\author{Alex Groce}
\affiliation{\institution{Northern Arizona University}\country{United States}}


%%
%% By default, the full list of authors will be used in the page
%% headers. Often, this list is too long, and will overlap
%% other information printed in the page headers. This command allows
%% the author to define a more concise list
%% of authors' names for this purpose.
\renewcommand{\shortauthors}{us folks}

%%
%% The abstract is a short summary of the work to be presented in the
%% article.
\begin{abstract}
Bitcoin is one of the most prominent distributed software systems in the world, and a key part of a potentially revolutionary new form of financial tool, cryptocurrency.  At heart, Bitcoin exists as a set of nodes running an implementation of the Bitcoin protocol.  This paper describes an effort to investigate and enhance the effectiveness of the Bitcoin Core implementation fuzzing effort.  The effort initially began as a query about how to escape \emph{saturation} in the fuzzing effort, but developed into a more general exploration once it was determined that saturation was largely illusory, a byproduct of the (then) fuzzing configuration.  This paper reports the process and outcomes of the two-week focused effort that emerged from that initial contact between Chaincode Labs and academic researchers.  That effort found no smoking guns indicating major test/fuzz weaknesses. However, it produced a large number of additional fuzz corpus entries to add to the Bitcoin QA assets, clarified some long-standing problems in OSS-Fuzz triage, increased the set of documented fuzzers used in Bitcoin Core testing, and ran the first mutation analysis of Bitcoin Core code, revealing opportunities for further improvement.  We contrast the Bitcoin Core transaction verification testing with that for other popular cryptocurrencies.  This paper provides an overview of the challenges involved in improving testing infrastructure, processess, and documentation for a highly visible open source target system, from both the state-of-the-art research perspective and the practical engineering perspective.  One major conclusion is that for well-designed fuzzing efforts, improvements to the \emph{oracle} side of testing, increasing invariant checks and assertions, may be the best route to getting more out of fuzzing.
\end{abstract}

\begin{CCSXML}
<ccs2012>
<concept>
<concept_id>10011007.10010940.10010992.10010998.10011001</concept_id>
<concept_desc>Software and its engineering~Dynamic analysis</concept_desc>
<concept_significance>500</concept_significance>
</concept>
<concept>
<concept_id>10011007.10011074.10011099.10011102.10011103</concept_id>
<concept_desc>Software and its engineering~Software testing and debugging</concept_desc>
<concept_significance>500</concept_significance>
</concept>
</ccs2012>
\end{CCSXML}

\ccsdesc[500]{Software and its engineering~Dynamic analysis}
\ccsdesc[500]{Software and its engineering~Software testing and debugging}

\keywords{fuzzing, saturation, test diversity, mutation analysis}


\maketitle


\section{Introduction}

Bitcoin~\cite{nakamoto2008bitcoin} is the most popular cryptocurrency, and one of the most visible (if controversial) ``new'' systems based on software to rise to prominence in the last decade.  As we write, while volatile, Bitcoin consistently has a market cap of over half a trillion dollars since January of 2021.  Due to its distributed, decentralized nature, Bitcoin in some sense is the sum of the operations of the code executed by many independent Bitcoin nodes, especially nodes that mine cryptocurrency.  Bitcoin Core (\url{https://github.com/Bitcoin/Bitcoin}) is by far the most popular implementation, and serves as a reference for all other implementations.  To a significant degree, the code of Bitcoin Core \emph{is} Bitcoin.  The main Bitcoin Core repo on GitHub has over 57,000 stars, and has been forked more than 30,000 times.

Because of its fame and the high monetary value of Bitcoins, the Bitcoin protocol and its implementations are a high-value target for hackers (or even nation states interested in controlling cryptocurrency developments).  Therefore, testing the code is of paramount importance, including extensive functional tests and aggressive \emph{fuzzing}.  This paper describes a focused effort to identify weaknesses in, and improve, the fuzzing of Bitcoin Core.




\balance

\bibliographystyle{ACM-Reference-Format}
\bibliography{bibliography}

\balance

\end{document}
