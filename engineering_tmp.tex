\section{Side Issues: Spurious Bugs, Fuzzer Mysteries, and AFL Stability}

A critical point about testing is that it is like other kinds of
software development.  In practice, while adding ``features'' or at
least optimizations of existing features --- in our context, adding
new fuzzers or cutting-edge methods such as swarm testing --- is
usually seen as the most interesting aspect of the work, the reality
is that developer/tester time is often spent on more mundane,
frustrating issues.

For example, in the Bitcoin Core effort, early attempts to apply
Eclipser and AFL ran into a problem:  at some point both fuzzers would
essentially stop working and begin producing a vast number of spurious
crashes.  By spurious crashes, we mean unreproducible crashes: inputs
that the fuzzer marks as causing a crash, but which, when executed in
a non-fuzzing context using the fuzz harness, do not crash.  Spurious
crashes are not unlike the pheonomenon of \emph{flaky tests} ~\cite{flaky}, one of
the banes of automated software testing.  A flaky test is a test that, for the same
snapshot of test code and code under test, sometimes
fails and sometimes passes.  In fact, arguably, spurious crashes in
fuzzing are \emph{flaky tests} where the flakiness is due to an
environmental dependence: the test fails when run under the fuzzer.
Flaky tests and spurious crashes consume signficant testing and
developer/test engineer resources.  During the very first discussions
with the Chaincode team, when Chaincode first ran AFL, some spurious
crashes were produced; these were eventually ignored.  This happens to
many spurious crashes, unlike flaky tests which manifest in standing
regression tests that often have to be either removed or otherwise
mitigated.  In fuzzing, simpy throwing away spurious crashes without
understanding them is a more feasible option.
OSS-Fuzz runs on Bitcoin Core, using AFL, also produced some spurious
crashes.

Unlike most spurious crashes, these were eventually (partly)
understood.  Because AFL and Eclipser, run more than a few hours,
inevitably reached as stage where they \emph{only} produced spurious
crashes (instead of doing any useful fuzzing), the problem had to be
understood, if these fuzzers were to be useful for serious fuzzing.
The first author discovered that the AFL and Eclipser problems were,
strictly speaking, not flaky or spurious at all.  The fuzz harnesses
all fail if there is insufficient storage space on {\tt tmp} to
execute the Bitcoin software.  At some point, this happens, even if
fuzzing begins with a large amount of free storage.  The problem is
that some (but not all) fuzz executions under AFL and Eclipser fail to
clean up {/tmp/test\_common\_Bitcoin Core}; each run that fails to
clean up leaves an 18-19 megabyte footprint.  Over the millions of
executions involved in a few hours of fuzzing, this inevitably fills
the space.

This problem was also (probably) causing the original AFL issues seen
by Chaincode, and (almost certainly, on examination of  the OSS-Fuzz
logs) the OSS-Fuzz failures that did not reproduce.  Multiple possible
solutions were discussed (at length) and proposed in a PR
(\url{https://github.com/bitcoin/bitcoin/pull/22472}), but as we write
none of these have been deemed successful.  Also, the underlying
reason why AFL and Eclipser sometimes fail to clean up {\tt /tmp} is
not understood; removing all threads did not fix the issue, though it
might be useful for other reasons.  The presence of threads may
explain the relatively low stability of AFL fuzzing on Bitcoin Core,
another issue raised during the 80 hour effort:
\url{https://github.com/bitcoin/bitcoin/issues/22551}.  Low stability
indicates that when an input is executed multiple times, it does not
always take the same path.  The worse the stability, the less useful signal
AFL is receiving from path coverage during fuzzing.  Bitcoin Core's
stability hovers around 80\%, possibly as a result of the presence of
threads in the code.  Even if the threads do not cause nondeterminisic
behavior of the Bitcoin Core system, from a functional point of view,
they may cause path coverage to vary more than AFL ``likes.''  It is
hard to guess if this has a serious adverse effect on fuzzing
performance.

One reason these issues have not been resolved is that they do not
affect the most effective (thus far) fuzzing approach.  Most new
coverage obtained on Bitcoin Core, and most bugs found, can be
credited to libFuzzer, which is not affected by the {\tt tmp} problem,
and at least fails to complain about stability, even if it is
affected.  At present, users of Eclipser on Bitcoin Core are advised
to perhaps apply one of the patches proposed in the open PR on the
problem, or to manually clean {\tt /tmp} in some other way.