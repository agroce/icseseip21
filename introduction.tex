\section{Introduction}

Bitcoin is the most popular cryptocurrency, and one of the most visible ``new'' systems based on software to rise to prominence in the last decade.  As we write, while volatile, Bitcoin consistently has a market cap of over half a trillion dollars since January of 2021.  Due to its distributed, decentralized nature, Bitcoin in some sense is the sum of the operations of the code executed by many independent Bitcoin nodes, especially nodes that mine cryptocurrency.  Bitcoin Core (\url{https://github.com/Bitcoin/Bitcoin}) is by far the most popular implementation, and serves as a reference for all other implementations.  To a significant degreee, the code of Bitcoin Core \emph{is} Bitcoin.  The main Bitcoin Core repo on GitHub has over 57,000 stars, and has been forked more than 30,000 times.

Because of its fame and the high monetary value of Bitcoins, the Bitcoin protocol and its implementations are a high-value target for hackers (or even nation states interested in controlling cryptocurrency developments).

Chaincode labs (\url{https://chaincode.com/}) is a private R\&D center based in Manhattan that exists solely to support and develop Bitcoin.  In March of 2021, Adam Jonas, the head of special projects at Chaincode, contacted the first author to discuss determining a strategy to improve the fuzzing of Bitcoin Core.  In particular, at the time, it seemed that the fuzzing was ``stuck'': neither code coverage nor found bugs were increasing with additional fuzzing time.  After some discussion, an 80 hour effort was determined as a reasonable scope for an external, research-oriented, look at the fuzzing effort.  Before that effort, conducted over the summer of 2021, began, the problem of saturation resolved itself.