\section{Conclusion: How Good is the Bitcoin Core Testing and Fuzzing?}

In a sense this is a fundamentally hard question to answer (software
engineering research has been trying to understand how to answer this question for
decades, after all): certainly,
an 80 hour effort by one fuzzing researcher, who was generally
familiar with blockchain and Bitcoin technology, but not at all
familiar with the Bitcoin Core implementation, in terms of high level
approach, testing infrastructure, or (of course) implementation details,
cannot provide anything like a definitive answer to that question.

The previous section discusses our limited effort to place the Bitcoin
Core test effort in the context of other cryptocurrencies, but the
limits of that evaluation make it hard to draw strong conclusions.
The complexities of the test efforts and the various ways the systems
could be compared make this a weak basis for evaluating Bitcoin Core itself.
Bitcoin Core has about 10,000 lines of fuzz harness code; Go-Ethereum
has about 74KLOC of fuzz harness code; is the fuzzing thus likely to
be 7 times better?  We don't claim to know.  Instead, we base our
evaluation on the effort to improve Bitcoin Core fuzzing, itself, plus
the fact that Bitcoin Core's coverage and mutation results seem very
likely to be at minimum competitive with those of other high-profile cryptocurrencies.

The initial fuzzing effort, at the time of contact, was certainly
operating in a suboptimal way,
and hence \emph{seemed} to be facing a difficult saturation problem.
However, the simplest of industrial-fuzzing-savvy advice (run the fuzzers longer, add the system
to OSS-Fuzz, try more fuzzers) was sufficient to make the
fundamentally sound fuzzing basis more useful, and the advice was
solicited and taken, very quickly.

Additional efforts to improve the fuzzing added some value in terms of
new corpus entries, and paved
the way for better documented, and more systematic, application of a
variety of fuzzers.  Fundamentally, however, these improvements were
relatively small compared to the overall scope of Bitcoin Core testing.
This was substantially due to the limited time-frame, but not due to either a
lack of expertise or serious attempts to improve fuzzing, given that
time-frame, we believe.

The basic cause of the limited improvement was that, at present,
Bitcoin Core has a well-designed and effective fuzzing infrastructure,
supported by intelligent manual augmentation of the more automated
aspects of the fuzzing, and in the context of an aggressive set of
functional tests with generally good code coverage.

\subsection{The Bitcoin Core Fuzzing Infrastructure}

\begin{figure}
  {\scriptsize
  \begin{code}
    FUZZ\_TARGET(parse\_script)
\{
    const std::string script\_string(buffer.begin(), buffer.end());
    try {
        (void)ParseScript(script\_string);
    } catch (const std::runtime\_error\&) {
    }
\}
\end{code}
}
\caption{Fuzz target for script parsing.}
\label{script}
\end{figure}
  

\begin{sloppypar}
The basic fuzzing infrastructure for Bitcoin resides in {\tt
  src/tests/fuzz}, which consists of about 200 fuzz target
implementations, ranging in size from a few lines to about 500 lines,
plus common infrastructure, primarily found in {\tt
  FuzzedDataProvider.h}.  The fuzzing implementation is compact, but
supports a common way to write fuzz targets that perform fuzz
generation of C++ generic and Bitcoin-specific data structures.  The
generators are designed to allow different fuzzers to serve as the
``back end'' of the system, as in DeepState.  This approach allows
some targets to be very compact, e.g., see Figure \ref{script}.

In two weeks, it is impossible to examine every fuzz target in detail,
much less understand how they interact with the targeted code, but
most inspected targets are well-designed, and in cases where targets
appear to be too generic, wasting time, as with {\tt
  process\_message}, there seems to be manual seeding that avoids the
problem of forcing fuzzers to generate too many magic strings.
\end{sloppypar}

One sign that the fuzzing is of reasonable quality is that the gap in code
coverage between fuzzing and high-quality
functional tests is not very large.  
For both fuzzing and functional tests, some missing coverage and some
unkilled mutants are almost
certainly (and stated to be such by Bitcoin Core team members to us)
redundant or defensive code to protect against ``impossible''
problems.  This is unsurprising in code intended to be highly reliable;
coverage of core file system functionality was only 89\% during a year of
extensive random testing of a file system developed for the Curiosity
Mars Rover performed by NASA/JPL's Laboratory for Reliable
Software~\cite{ICSEDiff}.  The missing coverage was mostly defensive,
redundant code, including fail-safe termination after (presumably
impossible without hardware failure)
assertion failures.

\subsection{Improving Bitcoin Core Fuzzing}

The lack of a significant coverage gap, and the existence of a huge
mutant-detection gap, for transaction verification, between the fuzzing and the functional tests
suggests the most promising method for improving the Bitcoin Core
fuzzing: manual, expert developer effort to improve the
oracles used by existing fuzz targets, or efforts to craft custom,
more restricted, fuzz targets with stronger oracles when this is not
feasible.

Essentially, the lack of a huge coverage gap suggests that failure to
explore the input space is not the biggest problem for the Bitcoin
Core fuzzing; there's significant room for improvement, especially
with some targets, but overall the coverage, especially for critical
code, is good.  What the mutation testing reveals is that many fuzz
executions that would be rejected by the standards of the functional
tests do not induce a crash.

Building fuzz harnesses with complex correctness checks is hard, of
course; the functional tests know exactly what inputs are being
provided to APIs, and can check for expected behavior.  Trying to
inject this kind of check into fuzz harnesses ranges from non-trivial
to effectively impossible (for some properties).  When
applicable, more generic, ``mathematical'' constraints such as are
used in property-based testing~\cite{ClaessenH00} can help, but these
are often hard to apply at the end-to-end level of a fuzz harness,
without turning the fuzz harness into a spaghetti code mess of
checks for various qualities of the inputs.  In the worst case, these
end up simply reflecting developer/tester notions that already found
their way into the code itself.  Differential testing~\cite{Differential} against other
implementations might also help here, but is probably best done at a
higher end-to-end level than inside these specific fuzz targets (this
is already done on some cryptographic elements of the code).

There is no
easy solution to this problem, but the most promising route is
probably to focus
on adding \emph{invariants and assertions} to the non-test code itself; these checks can be executed by
both functional and fuzz tests, and avoid the problem of duplicating
analysis of inputs.  At
present, the Bitcoin Core code has about 1,800 {\tt assert}
statements, scattered among  ~180KLOC of C and C++.  The resulting ratio
of about one assertion per 100 lines of code is not terrible, but is
at the lower limit of what many consider to be an acceptable assertion
ratio for critical code.
Given that Bitcoin Core defines at least 4,000 functions, the code obviously
doesn't meet the NASA/JPL proposal of having an average of two
assertions per function~\cite{holzmann2006power}.  There are only five assert
statements in the {\tt src/consensus} directory, which has about 500
lines of code and defines more than 10 functions, suggesting that the assertion ratio is low even for
critical code.

One conclusion of the 80 hour effort therefore, is that the most
effective way to improve fuzzing at present might be not to focus on
covering the code and state space, but to focus on increasing the
oracle power of \emph{all} Bitcoin Core testing.  Arguably, the
greatest weakness of traditional code coverage is that it focuses too
much attention on the ``input side'' of testing and too little on the
oracle side~\cite{barr2014oracle}, which is easier for even dedicated
testing efforts to neglect in the pursuit of covering every branch and
path.  