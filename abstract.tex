Bitcoin is one of the most prominent distributed software systems in the world, and a key part of a potentially revolutionary new form of financial tool, cryptocurrency.  At heart, Bitcoin exists as a set of nodes running an implementation of the Bitcoin protocol.  This paper describes an effort to investigate and enhance the effectiveness of the Bitcoin Core implementation fuzzing effort.  The effort initially began as a query about how to escape \emph{saturation} in the fuzzing effort, but developed into a more general exploration once it was determined that saturation was largely illusory, a byproduct of the (then) fuzzing configuration.  This paper reports the process and outcomes of the two-week focused effort that emerged from that initial contact between Chaincode labs and academic researchers.  That effort found no smoking guns indicating major test/fuzz weaknesses. However, it produced a large number of additional fuzz corpus entries to add to the Bitcoin QA assets, clarified some long-standing problems in OSSFuzz triage, increased the set of documented fuzzers used in Bitcoin Core testing, and ran the first mutation analysis of Bitcoin Core code, revealing opportunities for further improvement.  We contrast the Bitcoin Core transaction verification testing with similar tests for the most popular Ethereum and dogecoin implementations.  This paper provides an overview of the challenges involved in improving testing infrastructure, processess, and documentation for a highly visible open source target system, from both the state-of-the-art research perspective and the practical engineering perspective.