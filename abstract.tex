Bitcoin is one of the most prominent distributed software systems in the world, and a key part of a potentially revolutionary new form of financial tool, cryptocurrency.  At heart, bitcoin exists as a set of nodes running an implementation of the bitcoin protocol.  This paper describes an effort, requested by Chaincode labs, which exists to support and develop bitcoin, to investigate and enhance the effectiveness of the bitcoin core implementation fuzzing effort.  The effort initially began as a query about how to escape \emph{saturation} in the fuzzing effort, but developed into a more general exploration once it was determined that saturation was largely illusory, a byproduct of the (then) fuzzing configuration.  This paper reports the process and outcomes of the two-week focused effort that emerged from that initial contact between Chaincode labs and academic researchers.  While no smoking guns indicating serious issues were found, a large number of additional fuzz corpus entries were added to the bitcoin QA assets, some long-standing problems in OSSFuzz triage were clarified, the set of documented fuzzers was increased, and the first mutation analysis of bitcoin core code revealed opportunities for further improvement.  We also contrast the bitcoin core transaction verification testing with similar tests for the most popular Ethereum and dogecoin implementations.  This paper provides an overview of the challenges involved in building testing infrastructure, processess, and documentation for a highly visible open source target system, from both the state-of-the-art research perspective and the practical engineering perspective.